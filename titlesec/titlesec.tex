% +--------------------------------------------------+
% | Typeset this file to get the documentation.      |
% +--------------------------------------------------+
%
% Copyright (c) 1998-2016 by Javier Bezos.
% All Rights Reserved.
%
% This file is part of the titlesec distribution release 2.10.2
% -----------------------------------------------------------
%
% It may be distributed and/or modified under the
% conditions of the LaTeX Project Public License, either version 1.3
% of this license or (at your option) any later version.
% The latest version of this license is in
%   http://www.latex-project.org/lppl.txt
% and version 1.3 or later is part of all distributions of LaTeX
% version 2003/12/01 or later.
% 
% This work has the LPPL maintenance status "maintained".
% 
% The Current Maintainer of this work is Javier Bezos.

\def\fileversion{2.10.2}
\def\docdate{2016-03-21}

\documentclass[a4paper]{ltxguide}
\usepackage[sf,bf,compact,topmarks,calcwidth,pagestyles]{titlesec}
\usepackage{titletoc}
\def\gobble#1{}
\def\cs#1{\expandafter\gobble\string\\#1}
\makeatletter
\newenvironment{desc}
  {\if@nobreak
     \vskip-\lastskip
     \vspace*{-2.5ex}%
   \fi
   \decl}
  {\enddecl}
\makeatother

\usepackage{textcomp,pslatex}
\usepackage[linktocpage]{hyperref}

\title{The \textsf{titlesec}, \textsf{titleps} and \textsf{titletoc} 
Packages\footnote{The \textsf{titlesec} package is currently at 
version 2.10.2.  \copyright{} 1998--2016 Javier Bezos.
The \textsf{titletoc} package is currently at 
version 1.6.  The \textsf{titleps} package is currently at version 
1.1.1 \copyright{} 1999--2016 Javier Bezos.  All Rights 
Reserved.}}

\author{Javier Bezos\footnote{For bug reports, comments and
suggestions go to \href{http://www.tex-tipografia.com/contact.html}%
{\texttt{http://www.tex-tipografia.com/contact.html}}.  English
is not my strong point, so contact me when you find mistakes in the
manual.  Other packages by the same author: \textsf{gloss} (with
Jos\'e Luis D\'{\i}az), \textsf{enumitem, accents, tensind, esindex,
dotlessi, babeltools}.}}

\date{\docdate}

\widenhead{2.1pc}{0pc}
\titlelabel{\thetitle.\quad}

\renewpagestyle{plain}[\small\sffamily\slshape]{
  \footrule
  \setfoot{}{\usepage}{}}

\newpagestyle{myps}[\small\sffamily\slshape]{
  \headrule
  \sethead{Titlesec}{\sectiontitle}{\usepage}}
  
\pagestyle{myps}

\newcommand{\examplesep}{%
  \begin{center}%
    \rule{4pt}{4pt}%
  \end{center}}

\contentsmargin{0pt}
\titlecontents{section}[1.8pc]
  {\addvspace{3pt}\bfseries}
  {\contentslabel[\thecontentslabel.]{1.8pc}}
  {}
  {\quad\thecontentspage}

\titlecontents*{subsection}[1.8pc]
  {\small}
  {\thecontentslabel. }
  {}
  {, \thecontentspage}
  [.---][.]

\addtolength{\topmargin}{-3pc}
\addtolength{\textwidth}{6pc}
\addtolength{\oddsidemargin}{-2pc}
\addtolength{\textheight}{7pc}

\raggedright
\parindent1em
\parskip0pt

\begin{document}

\maketitle
\tableofcontents
\section{Introduction}

This package is essentially a replacement---partial or total---for the 
\LaTeX{} macros related with sections---namely titles, headers and 
contents.  The goal is to provide new features unavailable in current 
\LaTeX{}; if you just want a more friendly interface than that of 
standard \LaTeX{} but without changing the way \LaTeX{} works you may 
consider using \textsf{fancyhdr}, by Piet van Oostrum, \textsf{sectsty},
by Rowland McDonnell, and \textsf{tocloft}, by Peter Wilson, which you
can make pretty things with.\footnote{Since the sectioning commands 
are rewritten, their behaviour could be somewhat different 
in some cases.}

Some of the new features provided are:
\begin{itemize}
\item Different classes and ``shapes'' of titles, with tools for very 
fancy formats.  You can define different formats for left and right 
pages, or numbered and unnumbered titles, measure the width of the 
title, add a new section level, use graphics, and many more.  The 
Appendix shows a good deal of examples, so jump forward right now!

\item Headers and footers defined with no |\...mark| intermediates,
and perhaps containing top, first \emph{and} bot marks at the same time.
Top marks correctly synchronized with titles, without 
incompatibilities with the float mechanism. Decorative elements easily
added, including picture environments.

\item Pretty free form contents, with the possibility of grouping 
entries of different levels in a paragraph or changing the format
of entries in the middle of a document.
\end{itemize}
\textsf{Titlesec} works with the standard classes and with many
others, including the AMS ones, and it runs smoothly with
\textsf{hyperref}.\footnote{However, be aware the AMS classes
reimplement the original internal commands.  These changes will be
lost here.  The compatibility with \textsf{hyperref} has been tested
with \textsf{dvips}, \textsf{dvipdfm} and \textsf{pdftex} but it is an
unsupported feature.  Please, check your version of
\textsf{hyperref} is compatible with \textsf{titlesec}.}
Unfortunately, it is not compatible with \textsf{memoir}, which
provides its own tools with a limited subset of the features available
in \textsf{titlesec}.

As usual, load the package in the standard way with 
|\usepackage|.  Then, redefine the sectioning commands with the 
simple, predefined settings (see section ``Quick Reference'') or with 
the provided commands if you want more elaborate formats (see section 
``Advanced Interface.'')  In the latter case, you only need to 
redefine the commands you'll use.  Both methods are available at the
same time, but because |\part| is usually implemented in a
non-standard way, it remains untouched by the simple settings and
should be changed with the help of the ``Advanced Interface.''


\section{Quick Reference}
%~~~~~~~~~~~~~~~~~~~~~~

The easiest way to change the format is by means of a set of package 
options and a couple of commands. If you feel happy with the
functionality provided by this set of tools, you need not go
further in this manual. Just read this section and ignore the
subsequent ones.

\subsection{Format}

There are three option groups controlling font, size and align.  You 
need not set all of these groups, since a default is provided for each 
one; however, you must use at least an option from them if you want
this ``easy setup.'' 
\begin{desc}
|rm sf tt md bf up it sl sc|
\end{desc}
Select the corresponding family/series/shape.  Default is |bf|.  

\begin{desc}
|big medium small tiny|
\end{desc}
Set the size of titles.
Default is |big|, which gives the size of standard classes.
With |tiny|, sections (except chapters) are typed in the text
size. |medium| and |small| are intermediate layouts.

\begin{desc}
|raggedleft center raggedright|
\end{desc}

Control the alignment.

\subsection{Spacing}

\begin{desc}
|compact|
\end{desc}
This option is independent from those above and reduces the spacing
above and below the titles.

\subsection{Uppercase}

\begin{desc}
|uppercase|
\end{desc}

\fbox{2.9} Uppercases titles.  Depending on the class, it might not work in
\verb|\chapter| and \verb|\part|.

\subsection{Tools}

\begin{desc}
|\titlelabel{<label-format>}|
\end{desc}
Changes the label format in sections, subsections, etc. A
|\thetitle| command is provided which is respectively |\thesection|,
|\thesubsection|, etc. The default value in standard classes is
\begin{verbatim}
\titlelabel{\thetitle\quad}
\end{verbatim}
and you may add a dot after the counter simply with
\begin{verbatim}
\titlelabel{\thetitle.\quad}
\end{verbatim}
That was done in this document.

\begin{desc}
|\titleformat*{<command>}{<format>}|
\end{desc}

This command allows to change the |<format>| of a sectioning
command, as for example:
\begin{verbatim}
\titleformat*{\section}{\itshape}
\end{verbatim}

\section{Advanced Interface}
%~~~~~~~~~~~~~~~~~~~~~~~~

Two commands are provided to change the title format.  The first one
is used for the ``internal'' format, i.~e., shape, font, label\dots,
the second one defines the ``external'' format, i.~e., spacing before
and after, indentation, etc.  This scheme is intended to easy
definitions, since in most of cases you will want to modify either
spacing or format.\footnote{Information is ``extracted'' from the
class sectioning commands, except for chapter and part.  Standard
definitions with |\cs{@startsection}| are presumed---if sections have
been defined without that macro, arbitrary values for the format an
the spacing are provided, which you may change later.  (Sadly, there
is no way to catch the chapter or part formats, and one similar to
that of standard classes will be used.)} That redefines existing
sectioning commands, but does not create \emph{new} ones.  New
sectioning levels can be added with |\titleclass|, as described below,
and then their format can be set with the commands described here.

\subsection{Format}

A set of shapes is provided, which controls the basic distribution of
elements in a title. The available shapes are:
\begin{description}
\item[hang] is the default value, with a hanging label.  (Like the 
standard |\section|.)

\item[block] typesets the whole title in a block (a paragraph) without 
additional formatting.  Useful in centered titles\,\footnote{The label 
will be slightly displaced to the left if the title is two or more 
lines long and the \texttt{hang} shape is used, except with explicit 
|\string\\|.} and special formatting (including graphic tools such as 
|picture|, |pspicture|, etc.)

\item[display] puts the label in a separate paragraph. (Like the
standard |\chapter|.)

\item[runin] A run-in title, like the standard
  |\paragraph|.\footnote{Well, not quite. The title is first boxed to
  avoid some unexpected results if, for example, there is a
  \texttt{\string\color} between the title and the
  text. Unfortunately, due to an optimization done by \TeX{}
  discretionaries may be lost. I have found no solution, except using
  \textsf{luatex}, which works as one could expect. Anyway, if the
  title doesn't contain hyphen or dashes, this is not usually a real
  problem.}

\item[leftmargin] puts the title at the left margin.  Titles at the 
very end of a page will be moved to the next one and will not stick 
out in the bottom margin, which means large titles can lead to 
underfull pages.\footnote{However, floats following the title a couple 
of lines after will interfere with the page breaking used here and 
sometimes the title may stick out.} In this case you may increase the 
stretchability of the page elements, use |\raggedbottom| or use the 
package option |nobottomtitles| described below.  Since the mechanism 
used is independent from that of the margin pars, they can overlap.  
A deprecated synonymous is |margin|.

\item[rightmargin] is like |leftmargin| but at the right margin.

\item[drop] wraps the text around the title, provided the
first paragraph is longer than the title (if not, they overlap).
The comments in |leftmargin| also apply here. 

\item[wrap] is quite similar to drop.  The only difference is 
while the space reserved in drop for the title is fixed, in wrap is 
automatically readjusted to the longest line.  The limitations 
explained below related to |calcwidth| also apply here.

\item[frame] Similar to display, but the title will be framed.
\end{description}

Note, however, some shapes do not make sense in chapters and
parts.

\begin{desc}
|\titleformat{<command>}[<shape>]{<format>}{<label>}{<sep>}{<before-code>}[<after-code>]|
\end{desc}

Here
\begin{itemize}
\item |<command>| is the sectioning command to be redefined, i.~e., 
|\part|,
|\chapter|, |\section|, |\subsection|, |\subsubsection|, |\paragraph| 
or |\subparagraph|.

\item The paragraph shape is set by |<shape>|, whose possible
values are those described above.

\item |<format>| is the format to be applied to the whole
title---label and text.  This part can contain vertical material (and
horizontal with some shapes) which is typeset just after the space
above the title.

\item The label is defined in |<label>|.  You may leave it empty if
there is no section label at that level, but this is not recommended
because by doing so the number is not suppressed in the table of
contents and running heads.
 
\item |<sep>| is the horizontal separation between label and title 
body and must be a length (it must not be empty). This space is 
vertical in |display| shape; in |frame| it is the distance from text 
to frame. Both |<label>| and |<sep>| are ignored in starred versions 
of sectioning commands. If you are using |picture| and the like, set 
this parameter to 0 pt.

\item |<before-code>| is code preceding the title body. The very last 
command can take an argument, which is the title 
text.\footnote{Remember font size can be changed safely between 
paragraphs only, and changes in the text should be made local with 
a group; otherwise the leading might be wrong---too large or too small.}
However, with the package option \texttt{explicit} the title must
be given explicitly with |#1| (see below).

\item |<after-code>| is code following the title body. The typeset
material is in vertical mode with |hang|, |block| and |display|;
in horizontal mode with |runin| and |leftmargin| (\fbox{2.7} with the latter,
at the beginning of the paragraph). Otherwise is ignored.
\end{itemize}

\begin{desc}
|\chaptertitlename|
\end{desc}

It defaults to |\chaptername| except in appendices where it
is |\appendixname|. Use it instead of |\chaptername| when defining
a chapter.

\subsection{Spacing}

\begin{desc}
|\titlespacing*{<command>}{<left>}{<before-sep>}{<after-sep>}[<right-sep>]|
\end{desc}

The starred version kills the indentation of the paragraph 
following the title, except in |drop|, |wrap| and |runin| where this
possibility does not make sense.
\begin{itemize}
\item |<left>| increases the left margin, except in the |...margin|, 
and |drop| shape, where this parameter sets the title width, in 
|wrap|, the maximum width, and in |runin|, the indentation just before 
the title.  With negative value the title overhangs.\footnote{This 
parameter is not equal to |<indent>| of |\cs{@startsection}|, which 
doesn't work correctly.  With a negative value in the latter and if 
|<indent>| is larger than the label width, the first line of the title 
will start in the outer margin, as expected, but the subsequent lines 
will not; worse, those lines will be shortened at the right margin.}

\item |<before-sep>| is the vertical space before the title.

\item |<after-sep>| is the separation between title and text---vertical 
with |hang|, |block|, and |display|, and horizontal with |runin|, 
|drop|, |wrap| and |...margin|. By making the value negative, you may 
define an effective space of less than |\parskip|.\footnote{See 
Goossens, Mittelbach and Samarin: \textit{The \LaTeX{} Companion}, 
Reading, Addison Wesley, 1993, p.~25.}

\item The |hang|, |block| and |display| shapes have the 
possibility of increasing the |<right-sep>| margin with this optional 
argument.
\end{itemize}

If you dislike typing the full skip values, including the |plus| and 
|minus| parameters, an abbreviation |*|$n$ is provided. In the 
|<before-sep>| argument this is equivalent to $n$ |ex| with some 
stretchability and a minute shrinkability. In the
|<after-sep>| some stretchability (smaller) and no shrinkability.%
\footnote{They stand for $n$ times |1ex plus .3ex minus .06ex| and
|1ex plus .1ex|, respectively.} Thus, you can write
\begin{verbatim}
\titlespacing{\section}{0pt}{*4}{*1.5}
\end{verbatim}
The lengths |\beforetitleunit| and |\aftertitleunit| are used
as units in the |*| settings and you can change them if you do not like
the predefined values (or for slight changes in the makeup, for
example).

\textbf{Notes.} |\titlespacing| does not work with either |\chapter|
and |\part| unless you change its title format as well by means of
|\titleformat|, the simple settings, or |\titleclass|. Arguments in
|\titlespacing| must be dimensions;  |\stretch| includes a command and
hence raises an error.

\subsection{Spacing related tools}

These commands are provided as tools for |\titleformat| and 
|\titlespacing|.

\begin{desc}
|\filright  \filcenter  \filleft  \fillast  \filinner  \filouter|
\end{desc}

Variants of the |\ragged...| commands, with slight differences.  In
particular, the |\ragged...| commands kills the left and right spaces
set by |\titlespacing|.\footnote{Remember the package
\textsf{ragged2e} provides some additional commands for alignment,
too, like \texttt{\string\justifying}.} |\fillast| justifies the
paragraph, except the last line which is
centered.\footnote{Admittedly, a weird name, but it is short.} These
commands work in the |frame| label, too.

|\filinner| and |\filouter| are |\filleft| or |\filright| depending
on the page. Because of the asynchronous \TeX{} page 
breaking, these commands can be used in |\chapter| only.
If you want a general tool to set different formats depending
on the page, see ``Extended settings'' below.

\begin{desc}
|\wordsep|
\end{desc}

The inter-word space for the current font. 

\begin{desc}
|indentafter noindentafter| \quad (Package options)
\end{desc}

By-pass the settings for all of sectioning commands.%
\footnote{Formerly |indentfirst| and |nonindentfirst|, now
deprecated.}


\begin{desc}
|rigidchapters rubberchapters| \quad (Package options)
\end{desc}

With |rigidchapters| the space for chapter titles is always the same, 
and |<after-sep>| in |\titlespacing| does not mean the space from the 
bottom of the text title to the text body as described above, but from 
the \textit{top} of the text title, i.~e., |<before-sep>| $+$ 
|<after-sep>| now is a fixed distance from the top of the page body to 
the main text. The default is |rubberchapters| where |<after-sep>| is 
the separation between title and text as usual. Actually, the name is 
misleading because it applies not only to the default chapter, but to 
any title of top class. (More on classes below.)

\begin{desc}
|bottomtitles nobottomtitles nobottomtitles*|  \quad (Package options)
\end{desc}

If |nobottomtitles| is set, titles close to the bottom margin will 
be moved to the next page and the margin will be ragged.  The minimal 
space required in the bottom margin not to move the title is set 
(approximately) by
\begin{verbatim}
\renewcommand{\bottomtitlespace}{<length>}
\end{verbatim}
whose default value is |.2\textheight|.  A simple ragged bottom on the 
page before is obtained with a value of 0 pt.  |bottomtitles| is the 
default, which simply sets |\bottomtitlespace| to a negative value.

The |nobottomtitles*| option provides more accurate computations but
titles of |margin|, |wrap| or |drop| shapes could be badly 
placed. Usually, you should use the starred version.

\begin{desc}
|aftersep largestsep|  \quad (Package options)
\end{desc}

By default, when there are two consecutive titles the |<after-sep>| 
space from the first one is used between them.  Sometimes this is not 
the desired behaviour, especially when the |<before-sep>| space is much 
larger than the |<after-sep>| one (otherwise the default seems 
preferable).  With |largestsep| the largest of them is used.  
Default is |aftersep|.

\begin{desc}
|\\  \\*|\\
|pageatnewline|  \quad (Package option)
\end{desc}

\fbox{2.6} In version 2.6 and later, \verb|\\| does not allow a page
break and therefore is equivalent to \verb|\\*|.  Since I presume none
wants a page break inside a title, this has been made the default.  If
for some extrange reason you want to allow page breaks inside the
titles, use the package option \verb|pageatnewline|, which is provided
for backward compatibility.

\subsection{Rules}

The package includes some tools for helping in adding rules and other
stuff below or above the title. Since the margins in titles may be modified,
these macros take into account the local settings to place rules properly.
They also take into account the space used by marginal titles.

\begin{desc}
|\titleline[<align>]{<horizontal material>}|\\
|\titlerule[<height>]|\\
|\titlerule*[<width>]{<text>}|
\end{desc}

The |\titleline| command allows inserting a line, which may
contain text and other ``horizontal'' material. it is intended 
mainly for rules and leaders but in fact is also useful for other 
purposes.  The line has a fixed width and hence must be filled, i.e., 
|\titleline{CHAPTER}| produces an underfull box.  Here the optional 
|<align>| (|l|, |r| or |c|) helps, so that you simply type, say, 
|\titleline[c]{CHAPTER}|.%
\footnote{The default is the \texttt{s} parameter of the
\texttt{\cs{makebox}} command.}

Using |\titleline| in places where vertical material is not expected 
can lead to anomalous results.  In other words, you can use it in the 
|<format>| (always) and |<after-code>| (|hang|, |display| and |block|) 
arguments; and in the |display| shape at the very beginning of the 
|<before-code>| and |<label>| argument as well.  But try it out, because 
very likely it works in other places.

The |\titlerule| command, which is enclosed automatically in
|\titleline| if necessary, can be used to build rules and
fillers. The unstarred
version draws rules of height .4 pt, or |<height>| if present.
For example:
\begin{verbatim}
\titlerule[.8pt]%
\vspace{1pt}%
\titlerule
\end{verbatim}
draws two rules of different heights with a separation of
1 pt.

The starred version makes leaders with the |<text>|
repeated in boxes of its natural width. The width of the boxes
can be changed to |<width>|, but the first box remains with
its natural width so that the |<text>| is aligned to the left
and right edges of the space to be filled.

For instance, with
\begin{verbatim}
\titleformat{\section}[leftmargin]
  {\titlerule*[1pc]{.}%
   \vspace{1ex}%
   \bfseries}
  {... definition follows
\end{verbatim}
leaders spanning over both main text and title precede the section.

\begin{desc}
|calcwidth| \quad (Package option)
\end{desc}

The |wrap| shape has the capability of measuring the lines in the title 
to format the paragraph.  This capability may be extended to other 
three shapes---namely |display|, |block| and |hang|---with this 
package option.  The length of the longest line is returned in 
|\titlewidth|.\footnote{There are two further 
parameters, |\string\titlewidthfirst| and |\string\titlewidthlast|, 
which return the length of the first and last lines. There are not 
specific tools for using them, but you can assign their values to 
|\string\titlewidth| and then use |\string\titleline*|.}

As far as \TeX{} is concerned, any box is considered typeset material. 
If the box has been enlarged with blank space, or if conversely a box 
with text has been smashed, the value of |\titlewidth| may be wrong 
(as far as humans is concerned). The |hang| shape, for instance, uses 
internally such a kind of boxes, but in this case this behaviour is 
desired when the title is flushed right; otherwise the |block| shape 
produces better results. In other words, using boxes whose natural 
width has been overridden may be wrong.\footnote{Which include 
justified lines, whose interword spacing has been enlarged.} Further, 
some commands may confuse \TeX{} and stop parsing the title. But if 
you stick to text, |\\| and |\\[...]| (and it is very unlikely you 
might want something else), there will be no problems.

Another 
important point is the |<before-code>|, |<label>|, |<sep>|, and 
|<title>| parameters (but not |<after-code>|) are evaluated twice at local 
scope; if you increase a counter \emph{globally}, you are increasing 
it twice. In most of cases, placing the conflicting assignment in the 
|<after-code>| parameter will be ok, but alternativey you can use the 
following macro.

\begin{desc}
|\iftitlemeasuring{<true>}{<false>}|
\end{desc}

\fbox{2.9} When the title is being measured (first pass), the |<true>|
branch is used, and when the title is actually typeset (second pass) 
the |<false>| branch is used.

\begin{desc}
|\titleline*[<align>]{<horizontal material>}|
\end{desc}
A variant of |\titleline| to be used only with |calcwidth|.
The text will be enclosed first in a box of width |\titlewidth|; this box
will be in turn enclosed in the main box with the specified alignment.
There is no equivalent |\titlerule| and therefore you must enclose
it explicitly in a |\titleline*| if you want the |\titlewidth| to
be taken into account:
\begin{verbatim}
\titleline*[c]{\titlerule[.8pc]}
\end{verbatim}

\subsection{Page styles}

\fbox{2.8} You can assign a page style to levels of class |top| and
|page|, as well as the default chapter with the following command:%
\footnote{Named in the short-lived version 2.7 as 
\texttt{\string\titlepagestyle}.}
\begin{desc}
|\assignpagestyle{<command>}{<pagestyle>}|
\end{desc}
For example, to suppress the page number in chapters write:
\begin{verbatim}
\assignpagestyle{\chapter}{empty}
\end{verbatim}

\subsection{Breaks}

\begin{desc}
|\sectionbreak    \subsectionbreak     \subsubsectionbreak|\\
|\paragraphbreak  \subparagraphbreak   \<section>break|
\end{desc}

By defining these command with |\newcommand| different page breaks
could be applied to different levels. In those undefined, a penalty
with the internal value provided by the class is used (typically
$-300$). For instance,
\begin{verbatim}
\newcommand{\sectionbreak}{\clearpage}
\end{verbatim}
makes sections begin a new page. In some layouts, the space
above the title is preserved even if the section begins a new
page; that's accomplished with:
\begin{verbatim}
\newcommand{\sectionbreak}{%
  \addpenalty{-300}%
  \vspace*{0pt}}
\end{verbatim}

\fbox{2.6} \verb|\<section>break| is available in the \verb|top| class,
too.  Suitable values would be \verb|\cleardoublepage| (the default if
\verb|openright|) and \verb|\clearpage| (the default if
\verb|openany|).  Thus, you can override \verb|openright| by defining
\verb|\chapterbreak| as \verb|\clearpage|, provided its class has been
changed to \verb|top| (in this example, parts will continue with the
\verb|openright| setting).

\begin{desc}
|\chaptertolists|
\end{desc}

\fbox{2.6} If defined, the usual white space written to lists (ie,
List of Figures and List of Tables) is replaced by the code in this
command.  If you do not want the white space when a chapter begins,
define it to empty, i.e.,
\begin{verbatim}
\newcommand{\chaptertolists}{}
\end{verbatim}

This command is not a general tool to control
spacing in lists, and is available only in titles of top class, so
it will not work with the default chapters except if you change their
class (on the other hand, |\...tolists| can be used in any title whose
class is top).

\subsection{Other Package Options}

\begin{desc}
|explicit| \quad (Package option)
\end{desc}

\fbox{2.7} With it, the title is not implicit after |<before-code>| but
must be given explicitly with |#1| as in, for example:
\begin{verbatim} 
\titleformat{\section}
 {..}
 {\thesection}{..}{#1.}
\end{verbatim}
(Compare it with the example in section \ref{sec:dotafter}.) 

\begin{desc}
|newparttoc oldparttoc| \quad (Package options)
\end{desc}

Standard parts write the toc entry number in a non standard way.
You may change that with |newparttoc| so that \textsf{titletoc}
or a similar package can manipulate the entry. (That works only if
|\part| has been redefined.)

\begin{desc}
|clearempty| \quad (Package options)
\end{desc}

Modifies the behaviour of |\cleardoublepage| so that the |empty| page
style will be used in empty pages.

\begin{desc}
|toctitles| \quad (Package option)
\end{desc}

\fbox{2.6} Changes the behaviour of the optional argument in
sectioning titles so that it sets only the running heads and not the
TOC entries, which will be based on the full title.

\begin{desc}
|newlinetospace| \quad (Package option)
\end{desc}

\fbox{2.6} Replaces every occurrence of \verb|\\| or \verb|\\*| in
titles by a space in running heads and TOC entries.  This way, you
do not have to repeat the title just to remove a formatting command.

\begin{desc}
|notocpart*| \quad (Package option)
\end{desc}

\fbox{2.10.1} Long ago (by the year 2000) I decided for some reason
\verb|\part*| would behave like the AMS classes and therefore there
should be a contents entry for it. This is somewhat odd, indeed, but
the very fact is nobody has complained until now! On the other hand,
restoring the behaviour one could expect after 15 years doesn't seem a
good idea. A new page/part style in on the way, but for the moment
this option restores the standard behaviour.

\subsection{Extended Settings}
%~~~~~~~~~~~~~~~~~~~~~~~~~

The first argument of both |\titleformat| and |\titlespacing| has an 
extended syntax which
allows to set different formats depending on the context.\footnote{% 
The \textsf{keyval} package is required for making use of it.} This 
argument can be a list of key/value pairs in the form:
\begin{desc}
|<key>=<value>, <key>=<value>, <key>, <key>,...|
\end{desc}
Currently, only pages and unnumbered versions are taken care of,
besides the sectioning command name. Thus, the available keys are:
\begin{itemize}
\item |name|. Allowed values are |\chapter|, |\section|, etc.
\item |page|. Allowed values are |odd| or |even|.
\item |numberless|. A valueless key. it is not necessary unless you
want to set different numbered (without this key) and unnumbered
(with |numberless|) variants.
\end{itemize}
The basic form described above with the name of a sectioning
command, say
\begin{verbatim}
\titleformat{\section} ...
\end{verbatim}
is in fact an abbreviation for
\begin{verbatim}
\titleformat{name=\section} ...
\end{verbatim}

Let's suppose we'd like a layout with titles in the
outer margin. We might set something like
\begin{verbatim}
\titleformat{name=\section,page=even}[leftmargin]
  {\filleft\scshape}{\thesection}{.5em}{}

\titleformat{name=\section,page=odd}[rightmargin]
  {\filright\scshape}{\thesection}{.5em}{}
\end{verbatim}
Since the page information is written to the |aux| file, at
least two runs are necessary to get the desired result.

The ``number'' version is usually fine when generating unnumbered 
variants since removing the label is the only change required in most 
cases, but if you need some special formatting, there is the 
|numberless| key which defines an alternative version for sections 
without numbers (namely those with level below |secnumdepth|, in the 
front and back matters and, of course, the starred version). For 
instance
\begin{verbatim}
\titleformat{name=\section}{...% The normal definition follows
\titleformat{name=\section,numberless}{...% The unnumbered
% definition follows
\end{verbatim}
Neither |<label>| nor |<sep>| are ignored in |numberless|
variants.

These keys are available to both |\titleformat| and
|\titlespacing|. By using |page| in one (or both) of them, odd and
even pages will be formatted differently. Actually,
``even'' and ``odd'' are well established \LaTeX{} terms, but
misleading. In one side printing the ``odd'' pages refer
to ``even'' pages as well (cf.\@ |\oddsidemargin|.)

If you intend to create different odd/even \emph{and} 
different numbered/unnumbered versions, it is recommended defining
the four variants. 

If you remove the page specifier from a sectioning command you
must remove the |.aux| file.

\subsection{Creating new levels and changing the class}

While the shapes and the like modify the behaviour of titles related
to the surrounding text, title classes allow to change the generic
behaviour of them.  With the help of classes you may insert, say, a
new |subchapter| level between |chapter| and |section|, or creating a
scheme of your own.  \emph{Making a consistent scheme and defining all
of related stuff like counters, macros, format, spacing and, if there
is a TOC, TOC format is left to the responsibility of the user.} There
are three classes: |page| is like the book |\part|, in a single page,
|top| is like |\chapter|, which begins a page and places the title at
the top, and |straight| is intended for titles in the middle of
text.\footnote{There is an further class named |part| to emulate the
article |\cs{part}|, but you should not use it at all.  Use the
|straight| class instead.  Remember some features rely in these
classes and \textsf{titlesec} does not change by default the
definition of \texttt{\string\part} and \texttt{\string\chapter}.}

\begin{desc}
|\titleclass{<name>}{<class>}|\\
|\titleclass{<name>}{<class>}[<super-level-cmd>]|
\end{desc}

If you do not use the optional argument, you just change
the |<class>| of |<name>|. For example:
\begin{verbatim}
\titleclass{\part}{straight}
\end{verbatim}
makes |part| of |straight| class.

When the second form is used, the level number is the following of 
|<super-level-cmd>|. For example:
\begin{verbatim}
\titleclass{\subchapter}{straight}[\chapter]
\newcounter{subchapter}
\renewcommand{\thesubchapter}{\Alph{subchapter}}
\end{verbatim}
creates a level under chapter (some additional code is shown as well,
but you must add to it the corresponding |\titleformat| and 
|\titlespacing| settings).\footnote{Regarding counters, the
\textsf{remreset} package can be useful.}
If the chapter level is 0, then the subchapter one is 1; the levels 
below are increased by one (section is 2, subsection is 3, and so on).

There are two sectioning commands which perform some extra actions
depending of its name and ignoring the class:
\begin{itemize}
\item |\chapter| logs the string defined in |\chaptertitlename|
 and the matter is taken into account.
\item |\part| does not encapsulates the label in the toc entry,
except if you use the |newparttoc| option.
\end{itemize}

\begin{desc}
|loadonly| \quad (Package option)
\end{desc}

Let us suppose you want to create your sectioning commands from scratch.
This package option ignores any previous definitions, if any,
and hence removes the possibility of using the options described
in ``Quick Reference.'' Then you use the \textsf{titlesec}
tools, and define the corresponding counters and labels.

\begin{desc}
|\titleclass{<name>}[<start-level-num>]{<class>}|
\end{desc}

Here, the |<name>| title is considered the top level, with number
|<start-level-num>| (typically 0 or $-$1).  It should be used only
when creating sectioning commands from scratch with the help of
|loadonly|, and there must be exactly one (no more, no less)
declaration of this kind.  After it, the rest of levels are added as
explained above.

\section{Additional Notes}
%~~~~~~~~~~~~~~~~~~~~~~~~

This part describes briefly some \LaTeX{} commands, useful
when defining sectioning titles.

\subsection{Fixed Width Labels}

The |\makebox| command allows to use fixed width label, which
makes the left margin of the actual title (not the label) to lie
in the same place. For instance (only the relevant code is
provided):
\begin{verbatim}
\titleformat{\section}
  {..}
  {\makebox[2em]{\thesection}}{..}{..}
\end{verbatim}

See your \LaTeX{} manual for further reference on boxing commands.

\subsection{Starred Versions}
\label{s:starred}

Using sectioning commands in the starred version is strongly 
discouraged.  Instead, you can use a set of markup oriented commands 
which are easy to define and modify, if necessary.  Thus, you can test 
different layouts before choosing amongst them.

Firstly remember if you say
\begin{verbatim}
\setcounter{secnumdepth}{0}
\end{verbatim}
sections will be not numbered but they will be included in both toc
and headers.

Now, let's
suppose you want to include some sections with a special content;
for example, a section (or more) with exercises. We will use an
environment named |exercises| whose usage is:
\begin{verbatim}
\section{A section}
Text of a normal section.

\begin{exercises}
\section{Exercises A}
Some exercises

\section{Exercises B}
Some exercises
\end{exercises}
\end{verbatim}

The following definition suppresses numbers but neither toc lines
nor headers.
\begin{verbatim}
\newenvironment{exercises}
  {\setcounter{secnumdepth}{0}}
  {\setcounter{secnumdepth}{2}}
\end{verbatim}

The following one adds a toc line but headers will remain
untouched:
\begin{verbatim}
\newenvironment{exercises}
 {\setcounter{secnumdepth}{0}%
  \renewcommand\sectionmark[1]{}}
 {\setcounter{secnumdepth}{2}}
\end{verbatim}

The following one updates the headers but there will be
no toc line:
\begin{verbatim}
\newenvironment{exercises}
 {\setcounter{secnumdepth}{0}%
  \addtocontents{toc}{\protect\setcounter{tocdepth}{0}\ignorespaces}}
 {\setcounter{secnumdepth}{2}%
  \addtocontents{toc}{\protect\setcounter{tocdepth}{2}\ignorespaces}}
\end{verbatim}
(I find the latter a bit odd in this particular example; the
first and second options are more sensible. The |\ignorespaces|
is not very important, and you need not it unless there is
unwanted space in the toc.)

That works with standard classes, but if you are using
\textsf{fancyhdr} or \textsf{titlesec} to define headers you need 
further refinement to kill the section number. In \textsf{titlesec}
that's accomplished with |\ifthesection| (see below).

As you can see, there are no |\addcontentsline|, no
|\markboth|, no |\section*|, just logical structure. Of 
course you may change it as you wish; for example if you decide
these sections should be typeset in small typeface, include
|\small|, and if you realize you do not like that, remove it.

While the standard \LaTeX{} commands are easier and more
direct for simple cases, I think the proposed method above is
far preferable in large documents.

\subsection{Variants}

Let's suppose we want to mark some sections as ``advanced topics''
with an asterisk after the label.
The following code does the job:
\begin{verbatim}
\newcommand{\secmark}{}
\newenvironment{advanced}
  {\renewcommand{\secmark}{*}}
  {}
\titleformat{\section}
  {..}
  {\thesection\secmark\quad}{..}{..}
\end{verbatim}

To mark the sections write
\begin{verbatim}
\begin{advanced}
\section{...}
...
\end{advanced}
\end{verbatim}

That marks sections but not subsections. If you like being
redundant and marking the subsection level as well, you must
define it accordingly.

\subsection{Putting a Dot after the Section Title}
\label{sec:dotafter}

Today this styling is not used, but formerly it was fairly common.
The basic technique was described above, but here is a reminder:
\begin{verbatim}
\newcommand{\periodafter}[1]{#1.}
\titleformat{\section}
 {..}
 {\thesection}{..}{..\periodafter}
\end{verbatim}

If you had to combine this dot with some command (perhaps an
underlining), you can say:
\begin{verbatim}
\newcommand{\periodafter}[2]{#1{#2.}}
\titleformat{\section}
 {..}
 {\thesection}{..}{..\periodafter{\ul}} % \ul from soul package
\end{verbatim}

However, you might prefer the package option \texttt{explicit}.

\section{\textsf{titleps} and Page Styles}
%~~~~~~~~~~~~~~~~~~~

The \textsf{titleps} package provides tools for one-stage setting of
page styles (headlines and footlines).  A higher-level interface is
used, where the mark mechanism is hidden and there is no need to deal
with |\leftmark|s and |\rightmark|s -- just use a command or variable
registered as a ``mark'' as the expected value will be returned, i.e.,
those when the mark was emitted, either by a sectioning command or
explicitly with |\chaptermark|, |\sectionmark|, etc.  A simple
example, whose meaning should be obvious, is:
\begin{verbatim}
\newpagestyle{main}{
  \sethead[\thepage][\chaptertitle][(\thesection] % even
          {\thesection)}{\sectiontitle}{\thepage}} % odd
\pagestyle{main}
\end{verbatim}

Other features are:
\begin{itemize}
\item Working top marks, compatible with floats (unlike the standard 
|\topmark|, which does not work correctly in \LaTeX{}).
\item Access to top, first and bot marks in a single headline/footline
(e.g., the first and last section numbers).
\item Marks for more than 2 sectioning levels.
\item Simple (and not so simple) headrules and footrules.
\item Headlines and footlines for pages with floats.
\item Headlines and footlines for specific floats (a sort of 
|\thispagestyle| for floats).
\item Multiple sets of marks (named here \emph{markset}s and 
\textit{extra marks}).
\end{itemize}

It can be used without \textsf{titlesec}, but you will get most of 
it when used together.  To load it as a separate package, use the
customary \verb|\usepackage{titleps}|, but with \textsf{titlesec} you
have to load it with:
\begin{verbatim}
\usepackage[pagestyles]{titlesec}
\end{verbatim}

Please, read |titleps.pdf| (or typeset |titleps.tex|) for further 
information.

\section{Contents: The \textsf{titletoc} package}
% ~~~~~~~~~~~~~~~~~~~~~~~~~~~~~~~~~~~~~~~~~~~~~~

This package is a companion to the \textsf{titlesec} package and it 
handles
toc entries. However, it is an independent package and you can use
it alone. The philosophy is similar to that of \textsf{titlesec}---instead
of hooking the commands as defined by standard \LaTeX{} and classes,
there are new commands which you can format the toc entries with
in a generic way. This means you have to learn just
two new basic command and a couple of tools, no more, and you have access 
to
new features. Paragraph format
and fonts are set with commands like |\\|, 
|\makebox|,
|\large|, |\itshape|, and so on, and entries are not shaped in any
fashion because they are pretty free form.

The behaviour of entries
defined with \textsf{titletoc} are different at some points
to those created with the standard commands. In particular:
\begin{itemize}
\item Pages are never broken between entries if the first one is
of an higher level than the second one as, for instance, between
a section and a subsection. If both of them are of the same
level, the break is allowed, and if the first is lower than
the second, it is considered a good place for a page break.

\item The symbols in the leaders are not centered but flushed
right. That is usually more convenient.
\end{itemize}

I would like to note no attempt to handle tocs can be complete
because the standard \LaTeX{} commands write directly some formatting
commands which cannot be removed.  This is particularly important in
lists of figures and tables, and in the |\part| command.\footnote{But
some of these issues are fixed by \textsf{titlesec}.}

\subsection{A ten-minute guide to \textsf{titletoc}}

Toc entries are treated as rectangular areas where the text
and probably a filler will be written. Let's draw such an
area (of course, the lines themselves are not printed):
\setlength{\unitlength}{1cm}
\begin{center}
\begin{picture}(8,2.2)
\put(1,1){\line(1,0){6}}
\put(1,2){\line(1,0){6}}
\put(1,1){\line(0,1){1}}
\put(7,1){\line(0,1){1}}
\put(0,.7){\vector(1,0){1}}
\put(8,.7){\vector(-1,0){1}}
\put(0,.2){\makebox(1,.5)[b]{\textit{left}}}
\put(7,.2){\makebox(1,.5)[b]{\textit{right}}}
\end{picture}
\end{center}

The space between the left page margin and the left edge of
the area will be named |<left>|; similarly we have |<right>|.
You are allowed to modify the beginning of the first line and
the ending of the last line. For example by ``taking up'' both
places with |\hspace*{2pc}| the area becomes:
\begin{center}
\begin{picture}(8,2.2)
\put(1,1){\line(1,0){5.5}}
\put(6.5,1){\line(0,1){.5}}
\put(6.5,1.5){\line(1,0){.5}}
\put(1.5,2){\line(1,0){5.5}}
\put(1,1.5){\line(1,0){.5}}
\put(1.5,1.5){\line(0,1){.5}}
\put(1,1){\line(0,1){.5}}
\put(7,1.5){\line(0,1){.5}}
\put(0,.7){\vector(1,0){1}}
\put(8,.7){\vector(-1,0){1}}
\put(0,.2){\makebox(1,.5)[b]{\textit{left}}}
\put(7,.2){\makebox(1,.5)[b]{\textit{right}}}
\end{picture}
\end{center}
And by ``clearing'' space in both places with |\hspace*{-2pc}|
the area becomes:
\begin{center}
\begin{picture}(8,2.2)
\put(1,1){\line(1,0){6.5}}
\put(7.5,1){\line(0,1){.5}}
\put(7.5,1.5){\line(-1,0){.5}}
\put(.5,2){\line(1,0){6.5}}
\put(1,1.5){\line(-1,0){.5}}
\put(.5,1.5){\line(0,1){.5}}
\put(1,1){\line(0,1){.5}}
\put(7,1.5){\line(0,1){.5}}
\put(0,.7){\vector(1,0){1}}
\put(8,.7){\vector(-1,0){1}}
\put(0,.2){\makebox(1,.5)[b]{\textit{left}}}
\put(7,.2){\makebox(1,.5)[b]{\textit{right}}}
\end{picture}
\end{center}

If you have seen tocs, the latter should be familiar to you--
the label at the very beginning, the page at the very end:
\begin{verbatim}
    3.2  This is an example showing that toc
         entries fits in that scheme . . . .   4
\end{verbatim}

All you need is to put these elements in the right way.  If you have 
reserved the space with |\hspace*{-2pc}|, simply put a box 2 pc 
width 
containing the section label or page so that this space will be
retrieved; this layout is used so often that two commands are provided
which does that for you:
\begin{itemize}
\item |\contentslabel{<length>}| creates the space at the beginning 
and
   prints the section number.
\item |\contentspage| creates a space at the end of length |<right>|
   and prints the page number aligned at the right.
\end{itemize}

Now, we are about to show the three basic commands:

\begin{desc}
|\dottedcontents{<section>}[<left>]{<above-code>}|\\
|                {<label width>}{<leader width>}|
\end{desc}

Here:
\begin{itemize}
\item |<section>| is the section name without backslash: |part|,
  |chapter|, |section|, etc. |figure| and |table| are allowed, too.
  (The backlash is omitted because we are dealing with the concept
   and not the |\part|, |\section|, etc. macros themselves. 
  Furthermore, |figure| and |table| are environments.)

\item |<above-code>| is code for the global formatting of the entry.
  Vertical material is allowed. At this point the value of
  |\thecontentslabel| (see below) is known which enables you to
  take decisions depending on its value (with the help of
  the \textsf{ifthen} package). You may use the \textsf{titlesec}
  |\filleft|, |\filright|, |\filcenter| and |\fillast| commands. 

\item |<left>| even if bracketed is currently mandatory and it
  sets the left margin from the left page margin.
  
\item |<label width>| is the width of the space created for the label,
  as described  above.
  
\item |<leader width>| is the width of the box containing the char to
  be used as filler, as described below.
\end{itemize}

The definitions for section and subsection entries in the \textsf{book}
class are roughly equivalent to: 
\begin{verbatim}
\contentsmargin{2.55em}
\dottedcontents{section}[3.8em]{}{2.3em}{1pc}
\dottedcontents{subsection}[6.1em]{}{3.2em}{1pc}
\end{verbatim}

\begin{desc}
|\titlecontents{<section>}[<left>]{<above-code>}|\\
|              {<numbered-entry-format>}{<numberless-entry-format>}|\\
|              {<filler-page-format>}[<below-code>]|
\end{desc}

Here |<section>|, |<left>| and |<above-code>| like above, and
\begin{itemize}
\item |<numbered-entry-format>| is in horizontal mode and it will
  be used just before the entry title. As in |\titleformat|, the
  last command can take an argument with the title.

\item |<numberless-entry-format>| is like the above if there is, well,
  no label.

\item|<filler-page-format>| is self explanatory. Fillers are created
with the |\titlerule| command which is shared by that package and
\textsf{titlesec}. However, when used in this context its behaviour
changes a little to fit the needs of toc leaders.\footnote{For
\TeX{}nicians, the default |\cs{xleaders}| becomes 
|\cs{leaders}|.} You might prefer a |\hspace| instead.

\item And finally |<below-code>| is code following the entry for, say,
  vertical space.
\end{itemize}

When defining entries, use |\addvspace| if you want to add vertical
space, and |\\*| instead of |\\| for line breaks.

This command can be used in the middle of a document to change
the format of toc/lot/lof entries at any point. The new format is
written to the toc file and hence two runs are necessary to
see the changes.

\begin{desc}
|\contentsmargin{<right>}|
\end{desc}

The value set is used in all of sections. If you are wondering
why, the answer is quite simple: in most of cases the |<right>|
margin will be constant. However, you are allowed to change
it locally in the |<before-code>| arguments. Note as well that
the default space in standard classes does not leave room to
display boldfaced page number above 100 and therefore you
might want to set a larger margin with this command.

The \textsf{book} class formats section entries
similarly (but not equally) to:
\begin{verbatim}
\titlecontents{section}
              [3.8em] % ie, 1.5em (chapter) + 2.3em 
              {}
              {\contentslabel{2.3em}}
              {\hspace*{-2.3em}}
              {\titlerule*[1pc]{.}\contentspage}
\end{verbatim}
Compare this definition with that given above and you will
understand how |\dottedcontents| is defined.

Although standard classes use font dependent units (mainly em),
it is recommended using absolute ones (pc, pt, etc.) to ensure
they entries are aligned correctly.

\subsection{And more}

Strict typographical rules state full text lines shouldn't 
surpass the last dot of the leaders; ideally they should be aligned.  
Surprisingly enough, \TeX{} lacks of a tool for doing that 
automatically---when you fill a box with leading dots, they can be 
centered in the box with the |\cleaders| primitive , ``justified'' 
with |\xleaders| or aligned with the outermost enclosing box with 
|\leaders|, but there is no way to align them with the ``current'' 
margin.

So, the only way to get a fine layout is by hand.  To do , you can 
use the an optional argument in the |\contentsmargin| command whose 
syntax in full is the following:
\begin{desc}
|\contentsmargin[<correction>]{<right>}|
\end{desc}

The |<correction>| length is added to the |<right>| one in all of lines 
except the last one, where the leaders are placed.  For instance, if 
the text lines are 6 pt longer than the last dot, you should rewrite the 
|\contentsmargin| command to add a  |<correction>| of 6 pt.%
\footnote{Usefully,  many dvi previewers allow to get the coordinates of
the pointed location.}  Unlike the standard \LaTeX{} tools, the 
|\titlerule*| command has been designed so that the |<correction>| 
will have the minimum value possible.

\begin{desc}
|\thecontentslabel  \thecontentspage|
\end{desc}

Contains the text with the label and the page with no additional
formatting, except  written by the class.

\begin{desc}
|\contentslabel[<format>]{<space>}|\\
|\contentspage[<format>]|
\end{desc}

As described above, but with different |<format>|s. The defaults are
just |\thecontentslabel| and |\thecontentspage|, respectively.

\begin{desc}
|\contentspush{<text>}|
\end{desc}

Prints the |<text>| and increases
|<left>| by the width of |<text>|. It is similar to
the hang shape of \textsf{titlesec}.

\begin{desc}
|\titlecontents*{<section>}[<left>]{<above-code>}|\\
|               {<numbered-entry-format>}{<numberless-entry-format>}|\\
|               {<filler-page-format>}[<separator>]|\\[3pt]
|            |\textit{or ...}|{<filler-page-format>}[<separator>][<end>]|\\
|            |\textit{or ...}|{<filler-page-format>}[<begin>][<separator>][<end>]|
\end{desc}

This starred version groups the entries in a single paragraph.
The |<separator>| argument is the separator between entries, and
there is a further optional argument with an ending
punctuation.  For example, this document sets:
\begin{verbatim}
\titlecontents*{subsection}[1.5em]
  {\small}
  {\thecontentslabel. }
  {}
  {, \thecontentspage}
  [.---][.]
\end{verbatim}
whose result is showed in the contents at the very beginning of
this document. Note  the paragraph format must be written in
the |<above-code>| argument.

Let us explain how the optional arguments works.  First note the number
of them determines their meaning---since there should be a separator
between entries this one is always present; on the other hand,
|<begin>| is rarely used and hence it has the lowest ``preference.''
The simplest case is when the titles are of the same level; in this
case the |<sepatator>| and the |<end>| parameters (which default to
empty) are inserted between consecutive entries and at the end of the
block, respectively.  |<before-code>| is executed just once at the
very beginning of the block and its declarations are local to the
whole set of entries.

Now suppose we want to group entries of two levels; in this 
case a nesting principle applies.  To fix ideas, we will use section 
and subsection.  When a subsection entry begins after a section one, 
|<before-code>| is executed and |<begin>| of subsection is 
inserted, which should contain text format only.  Then subsections are 
added inserting separators as explained above.  When a section 
arrives, the ending punctuation of subsection and the separator of 
section is added (except if the block is finished by a subsection, 
where the ending of section is added instead).  We said ``after a 
section'' because a subsection never begins a block.\footnote{In rare 
cases that could be necessary, yet.} The 
subsection entries are nested inside the section ones, and 
declarations are again local.

An example will illustrate that.
\begin{verbatim}
\titlecontents*{section}[0pt]
  {\small\itshape}{}{}
  {}[ \textbullet\ ][.]

\titlecontents*{subsection}[0pt]
  {\upshape}{}{}
  {, \thecontentspage}[ (][. ][)]
\end{verbatim}
produces something similar to:
$$\begin{minipage}{\textwidth}
\small\itshape The first section \textbullet\ The second one  \textbullet\ 
The third one {\upshape(A subsection in it, 1. Another, 2)} \textbullet\ A
fourth section {\upshape(A subsection in it, 1. Another, 2)}.
\end{minipage}$$

\begin{desc}
|\contentsuse{<name>}{<ext>}|
\end{desc}

Makes \textsf{titletoc} aware of the existence of a contents file with 
|<ext>| extension. Mainly, it makes sure the command 
|\contentsfinish| is added at the end of the corresponding 
contents (and which must be added at the end of tocs made by hand). 
The package performs
\begin{verbatim}
\contentsuse{figure}{lof}
\contentsuse{table}{lot}
\end{verbatim}

% \begin{desc}
% |\titlelevels{<top>}{<level-list>}|
% \end{desc}
% 
% If you are not using \textsf{titlesec}, this command modifies
% the list of level names. Only necessary if you have been devised
% your own scheme of titles.

\begin{desc}
|leftlabels  rightlabels| \quad (Package options)
\end{desc}

These package options set how the labels are aligned in 
|\contentslabel|.
Default is |rightlabels|. With |leftlabels| the default |<format>| for
|\contentslabel| becomes |\thecontenstlabel\enspace|.

\begin{desc}
|dotinlabels| \quad (Package option)
\end{desc}

With this package option, a dot is added after the label in 
|\contentslabel|.

\subsection{Partial TOC's}

\begin{desc}
|\startcontents[<name>]|
\end{desc}

At the point where this command is used, a partial toc begins (note
the document doesn't require a |\tableofcontents| for partial tocs to
work).  The |<name>| argument allows different sets of tocs and it
defaults to |default|.  These sets may be intermingled, but usually
will be nested.  For example, you may want two kinds of partial tocs:
by part and by chapter (besides the full toc, of course).  When a part
begins, write |\startcontents[parts]|, and when a chapter
|\startcontents[chapters]|.  This way a new toc is started at each
part and chapter.\footnote{\emph{All} partial tocs are stored in a
single file with extension |.ptc|.}

\begin{desc}
|\stopcontents[<name>]|\\
|\resumecontents[<name>]|
\end{desc}

Stops the partial toc of |<name>| kind, which may be resumed.
Since partial contents are stopped by |\startcontents| if necessary,
those macros will not be used very often.

\begin{desc}
|\printcontents[<name>]{<prefix>}{<start-level>}{<toc-code>}|
\end{desc}

Print the current partial toc of |<name>| kind. The format
of the main toc entries are used, except if there is a |<prefix>|.
In such a case, the format of |<prefix><level>| is used, provided
it is defined. For example, if prefix is |l| and the format of
|lsection| is defined, then this definition will be used; otherwise,
the format is that of |section|. The |<start-level>| parameter sets the
top level of the tocs---for a part toc it would be |0| (chapter), for a
chapter toc |1| (section), and so on. Finally, |<toc-code>| is
local code for the current toc; it may be used to change the
|tocdepth| value or |\contentsmargin|, for instance.

A simple usage might look like (provided you are using 
\textsf{titlesec} as well):
\begin{verbatim}
\titleformat{\chapter}[display]
  {...}{...}{...}  % Your definitions come here
  [\vspace*{4pc}%
   \startcontents
   \printcontents{l}{1}{\setcounter{tocdepth}{2}}]
   
\titlecontents*{lsection}[0pt]
  {\small\itshape}{}{}
  {}[ \textbullet\ ][.]
\end{verbatim}
The included entries are those in levels 1 to 2 inclusive (i.e., 1 
and 2).

\subsection[Partial lists]{Partial lists \normalfont\normalsize\fbox{2.6}}

You may want to create partial LOFs and LOTs. The systax is similar to
that of partial TOCs and what was said for them can be applied here.
The commands are:
\begin{desc}
|\startlist[<name>]{<list>}|\\
|\stoplist[<name>]{<list>}|\\
|\resumelist[<name>]{<list>}|\\
|\printlist[<name>]{<list>}{<prefix>}{<toc-code>}|
\end{desc}

Here |<list>| is either |lof| or |lot|.  Note as well |\printlist|
does not have the |<start-level>| argument, because figures and tables
have not levels.  Currently, only those two float lists are supported,
but in a future release support for more kinds of float lists will be
added.  Unfortunately, many classes write some formatting commands to
these lists (more precisely, \verb|\addvspace|s in chapters); I'm
still not sure how to remove these commands without removing as well
others which can be wanted, but for the time being a quick trick to
remove these spaces is to redefine \verb|\addvspace| in the
|<toc-code>| with |\renewcommand\addvspace[1]{}|.

\subsection{Examples}

\begin{verbatim}
\titlecontents{chapter}
              [0pt]
              {\addvspace{1pc}%
               \itshape}%
              {\contentsmargin{0pt}%
               \bfseries
               \makebox[0pt][r]{\huge\thecontentslabel\enspace}%
               \large}
              {\contentsmargin{0pt}%
               \large}
              {\quad\thepage}
              [\addvspace{.5pc}]
\end{verbatim}

The chapter number is out at the edge of the page margin, in a font
larger than the font of the title. If the chapter lacks of number
(because, say, it is the preface or the bibliography) it is not
boldfaced. The page number follows the title without fillers, but
after an em-space.

\begin{verbatim}
\titlecontents{chapter}
              [3pc]
              {\addvspace{1.5pc}%
               \filcenter}
              {CHAPTER \thecontentslabel\\*[.2pc]%
               \huge}
              {\huge}
              {}  % That is, without page number
              [\addvspace{.5pc}]
\end{verbatim}
              
The chapter title is centered with the chapter label on top
of it. There is no page number.

\subsection{Inserting a figure in the contents}

The |\addtocontents| command is still available and you may use
it to perform special operation, like inserting a figure just before
or after of an entry. Sadly, fragile
arguments are not allowed and writing complex code could be a mess.
The trick is to define a command to perform the required operations
which in turn is written with |\protect|.

Let's suppose we want to insert a figure before an entry.
\begin{verbatim}
\newcommand{\figureintoc}[1]{
  \begin{figure}
    \includegraphics{#1}%
  \end{figure}}
\end{verbatim}
makes the dirty work.

In the place where a figure is inserted write:
\begin{verbatim}
\addtocontents{\protect\figureintoc{myfig}}
\end{verbatim}

\subsection{Marking entries with asterisks}

Let's now resume a problem explained in relation with
\textsf{titlesec}: marking sections with asterisks to
denote an ``advanced topic'' unless the star should
be printed in the toc as well. Here is the code:
\begin{verbatim}
\newcommand{\secmark}{}
\newcommand{\marktotoc}[1]{\renewcommand{\secmark}{#1}}
\newenvironment{advanced}
  {\renewcommand{\secmark}{*}%
   \addtocontents{toc}{\protect\marktotoc{*}}}
  {\addtocontents{toc}{\protect\marktotoc{}}}
\titleformat{\section}
  {..}
  {\thesection\secmark}{..}{..}
\titlecontents{section}[..]{..}
  {\contentslabel[\thecontentslabel\secmark]{1.5pc}}{..}{..}
\end{verbatim}

\section{The \textsf{titlesec} philosophy}

Once you have read the documentation it should be clear this
is not a package for the casual user who likes the standard
layout and wants to make simple changes. This is a tool for the
serious typographer who has a clear idea of what layout wants
and do not have the skill to get it. No attempt is made to improve
your taste in section formatting.

\section{Appendix}

The following examples will be illustrative. In this part, the
|\parskip| is 0 pt.

\begingroup

\addtocontents{toc}{\protect\setcounter{tocdepth}{-1}\ignorespaces}
\setlength{\parskip}{0pt}

\examplesep

\titleformat{\section}[block]
  {\normalfont\bfseries\filcenter}{\fbox{\itshape\thesection}}{1em}{}

\section[Appendix]{This is an 
example of the section command defined below and, what's more, this 
is an example of the section command defined below}

\begin{verbatim}
\titleformat{\section}[block]
  {\normalfont\bfseries\filcenter}{\fbox{\itshape\thesection}}{1em}{}
\end{verbatim}

\examplesep

\titleformat{\section}[frame]
  {\normalfont}
  {\filright
   \footnotesize
   \enspace SECTION \thesection\enspace}
  {8pt}
  {\Large\bfseries\filcenter}

\section[Appendix]{A framed title}

\begin{verbatim}
\titleformat{\section}[frame]
  {\normalfont}
  {\filright
   \footnotesize
   \enspace SECTION \thesection\enspace}
  {8pt}
  {\Large\bfseries\filcenter}
\end{verbatim}

\examplesep

\titleformat{\section}
  {\titlerule
   \vspace{.8ex}%
   \normalfont\itshape}
  {\thesection.}{.5em}{}

\section[Appendix]{A Ruled Title}

\begin{verbatim}
\titleformat{\section}
  {\titlerule
   \vspace{.8ex}%
   \normalfont\itshape}
  {\thesection.}{.5em}{}
\end{verbatim}

\examplesep

\titleformat{\section}[block]
  {\normalfont\sffamily}
  {\thesection}{.5em}{\titlerule\\[.8ex]\bfseries}
  
\section[Appendix]{Another Ruled Title}

\begin{verbatim}
\titleformat{\section}[block]
  {\normalfont\sffamily}
  {\thesection}{.5em}{\titlerule\\[.8ex]\bfseries}
\end{verbatim}

\examplesep

\titleformat{\section}[block]
  {\filcenter\large
   \addtolength{\titlewidth}{2pc}%
   \titleline*[c]{\titlerule*[.6pc]{\tiny\textbullet}}%
   \addvspace{6pt}%
   \normalfont\sffamily}
  {\thesection}{1em}{}
\titlespacing{\section}
  {5pc}{*2}{*2}[5pc]

\section[Appendix]{The length of the ``rule'' above
  is that of the longest line in this title increased by
  two picas}

\leavevmode

\section[Appendix]{This one is shorter}

\begin{verbatim}
\titleformat{\section}[block]
  {\filcenter\large
   \addtolength{\titlewidth}{2pc}%
   \titleline*[c]{\titlerule*[.6pc]{\tiny\textbullet}}%
   \addvspace{6pt}%
   \normalfont\sffamily}
  {\thesection}{1em}{}
\titlespacing{\section}
  {5pc}{*2}{*2}[5pc]
\end{verbatim}

\examplesep

\titleformat{\section}[display]
  {\normalfont\fillast}
  {\scshape section \oldstylenums{\thesection}}
  {1ex minus .1ex}
  {\small}
\titlespacing{\section}
  {3pc}{*3}{*2}[3pc]

\section[Appendix]{This is an example of the section
command defined below
and, what's more, this is 
an example of the section command defined below. Let us repeat it.
This is an example of the section command defined below
and, what's more, this is 
an example of the section command defined below}

\begin{verbatim}
\titleformat{\section}[display]
  {\normalfont\fillast}
  {\scshape section \oldstylenums{\thesection}}
  {1ex minus .1ex}
  {\small}
\titlespacing{\section}
  {3pc}{*3}{*2}[3pc]
\end{verbatim}

\examplesep

\titleformat{\section}[runin]
  {\normalfont\scshape}
  {}{0pt}{}
\titlespacing{\section}
  {\parindent}{*2}{\wordsep}
  
\section*{This part is the title itself}
and this part is the section body\ldots

\begin{verbatim}
\titleformat{\section}[runin]
  {\normalfont\scshape}
  {}{0pt}{}
\titlespacing{\section}
  {\parindent}{*2}{\wordsep}
\end{verbatim}

\examplesep

\titleformat{\section}[wrap]
  {\normalfont\fontseries{b}\selectfont\filright}
  {\thesection.}{.5em}{}
\titlespacing{\section}
  {12pc}{1.5ex plus .1ex minus .2ex}{1pc}

\section[Appendix]{A Simple Example of the
  ``wrap'' Section Shape}

Which is followed by some text to show the result.  Which is followed 
by some text to show the result.  Which is followed by some text to 
show the result.  Which is followed by some text to show the result.  
Which is followed by some text to show the result.  Which is followed 
by some text to show the result.  Which is followed 
by some text to show the result.

\section[Appendix]{And another}

Note how the text wraps the title and the space reserved to it is
readjusted automatically. And it is followed by some text to 
show the result.  Which is followed by some text to show the result.

\begin{verbatim}
\titleformat{\section}[wrap]
  {\normalfont\fontseries{b}\selectfont\filright}
  {\thesection.}{.5em}{}
\titlespacing{\section}
  {12pc}{1.5ex plus .1ex minus .2ex}{1pc}
\end{verbatim}

\examplesep

\titleformat{\section}[runin]
  {\normalfont\bfseries}
  {\S\ \thesection.}{.5em}{}[.---]
\titlespacing{\section}
  {\parindent}{1.5ex plus .1ex minus .2ex}{0pt}

\section[Appendix]{Old-fashioned runin title}

Of course, you would prefer just a dot after the title. In
this case the optional argument should be |[.]| and the space
after a sensible value (1em, for example).

\begin{verbatim}
\titleformat{\section}[runin]
  {\normalfont\bfseries}
  {\S\ \thesection.}{.5em}{}[.---]
\titlespacing{\section}
  {\parindent}{1.5ex plus .1ex minus .2ex}{0pt}
\end{verbatim}

\examplesep

\titleformat{\section}[leftmargin]
  {\normalfont
   \titlerule*[.6em]{\bfseries .}%
   \vspace{6pt}%
   \sffamily\bfseries\filleft}
  {\thesection}{.5em}{}
\titlespacing{\section}
  {4pc}{1.5pc plus .1ex minus .2ex}{1pc}

\section*{Example of margin section}

Which is followed by some text to show the result.
But do not stop reading, because the following example illustrates how
to take advantage of other packages. The last command in the last
argument can take an argument, which is the title with no
additional command inside it. We just give the code, but you
may try it by yourself. Thus, with the \textsf{soul} package
you may say
\begin{verbatim}
\newcommand{\secformat}[1]{\MakeLowercase{\so{#1}}}
   % \so spaces out letters
\titleformat{\section}[block]
  {\normalfont\scshape\filcenter}
  {\thesection}
  {1em}
  {\secformat}
\end{verbatim}

The margin title above was defined:
\begin{verbatim}
\titleformat{\section}[leftmargin]
  {\normalfont
   \titlerule*[.6em]{\bfseries.}%
   \vspace{6pt}%
   \sffamily\bfseries\filleft}
  {\thesection}{.5em}{}
\titlespacing{\section}
  {4pc}{1.5ex plus .1ex minus .2ex}{1pc}
\end{verbatim}

\examplesep

The following examples are intended for chapters. However, this
document lacks of |\chapter| and are showed using |\section|s
with slight changes.

\titlespacing{\section}{0pt}{*4}{*4}
\titleformat{\section}[display]
  {\normalfont\Large\filcenter\sffamily}
  {\titlerule[1pt]%
   \vspace{1pt}%
   \titlerule
   \vspace{1pc}%
   \LARGE\MakeUppercase{chapter} \thesection}
  {1pc}
  {\titlerule
   \vspace{1pc}%
   \Huge}

\section[Appendix]{The Title}

\begin{verbatim}
\titleformat{\chapter}[display]
  {\normalfont\Large\filcenter\sffamily}
  {\titlerule[1pt]%
   \vspace{1pt}%
   \titlerule
   \vspace{1pc}%
   \LARGE\MakeUppercase{\chaptertitlename} \thechapter}
  {1pc}
  {\titlerule
   \vspace{1pc}%
   \Huge}
\end{verbatim}
   
\examplesep

\def\thesection{\Roman{section}}
\titleformat{\section}[display]
  {\bfseries\Large}
  {\filleft\MakeUppercase{chapter} \Huge\thesection}
  {4ex}
  {\titlerule
   \vspace{2ex}%
   \filright}
  [\vspace{2ex}%
   \titlerule]

\section[Appendix]{The Title}

\begin{verbatim}
\renewcommand{\thechapter}{\Roman{chapter}}
\titleformat{\chapter}[display]
  {\bfseries\Large}
  {\filleft\MakeUppercase{\chaptertitlename} \Huge\thechapter}
  {4ex}
  {\titlerule
   \vspace{2ex}%
   \filright}
  [\vspace{2ex}%
   \titlerule]
\end{verbatim}

\addtocontents{toc}{\protect\setcounter{tocdepth}{2}\ignorespaces}
\setcounter{section}{9}
\endgroup

\bigskip 

\subsection{A full example}

Now an example of a complete title scheme follows.

\begin{verbatim}
\documentclass[twoside]{report}
\usepackage[sf,sl,outermarks]{titlesec}

% \chapter, \subsection...: no additional code

\titleformat{\section}
  {\LARGE\sffamily\slshape}
  {\thesection}{1em}{}
\titlespacing{\section}
  {-6pc}{3.5ex plus .1ex minus .2ex}{1.5ex minus .1ex}

\titleformat{\paragraph}[leftmargin]
  {\sffamily\slshape\filright}
  {}{}{}
\titlespacing{\paragraph}
  {5pc}{1.5ex minus .1 ex}{1pc}

% 5+1=6, ie, the negative left margin in section

\widenhead{6pc}{0pc}
  
\renewpagestyle{plain}{}

\newpagestyle{special}[\small\sffamily]{
   \headrule
   \sethead[\usepage][\textsl{\chaptertitle}][]
           {}{\textsl{\chaptertitle}}{\usepage}}
               
\newpagestyle{main}[\small\sffamily]{
   \headrule
   \sethead[\usepage][\textsl{\thechapter. \chaptertitle}][]
           {}{\textsl{\thesection. \sectiontitle}}{\usepage}}

\pagestyle{special}

\begin{document}

---TOC

\pagestyle{main}

---Body

\pagestyle{special}

---Index
\end{document}
\end{verbatim}

\subsection{Standard Classes}

Now follows, for your records, how sectioning commands of standard 
classes could be defined.
\begin{verbatim}
\titleformat{\chapter}[display]
  {\normalfont\huge\bfseries}{\chaptertitlename\ \thechapter}{20pt}{\Huge}
\titleformat{\section}
  {\normalfont\Large\bfseries}{\thesection}{1em}{}
\titleformat{\subsection}
  {\normalfont\large\bfseries}{\thesubsection}{1em}{}
\titleformat{\subsubsection}
  {\normalfont\normalsize\bfseries}{\thesubsubsection}{1em}{}
\titleformat{\paragraph}[runin]
  {\normalfont\normalsize\bfseries}{\theparagraph}{1em}{}
\titleformat{\subparagraph}[runin]
  {\normalfont\normalsize\bfseries}{\thesubparagraph}{1em}{}

\titlespacing*{\chapter}      {0pt}{50pt}{40pt}
\titlespacing*{\section}      {0pt}{3.5ex plus 1ex minus .2ex}{2.3ex plus .2ex}
\titlespacing*{\subsection}   {0pt}{3.25ex plus 1ex minus .2ex}{1.5ex plus .2ex}
\titlespacing*{\subsubsection}{0pt}{3.25ex plus 1ex minus .2ex}{1.5ex plus .2ex}
\titlespacing*{\paragraph}    {0pt}{3.25ex plus 1ex minus .2ex}{1em}
\titlespacing*{\subparagraph} {\parindent}{3.25ex plus 1ex minus .2ex}{1em}
\end{verbatim}

\subsection{Chapter Example}

A final example shows how to take advantage of the |picture|
environment for fancy sectioning formats. Even with the simple
tools provided by standard \LaTeX{} you may create impressive
titles but you may devise more elaborated
ones with, for instance, |pspicture| (\textsf{PSTricks}
package) or by including graphics created with the help of external
programs.

\begin{verbatim}
\usepackage[dvips]{color}
\usepackage[rigidchapters,explicit]{titlesec}
    
\DeclareFixedFont{\chapterfont}{T1}{phv}{bx}{n}{11cm}

\titlespacing{\chapter}{0pt}{0pt}{210pt}
% Most of titles have some depth. The total space is
% a bit larger than the picture box.

\titleformat{\chapter}[block]
  {\begin{picture}(330,200)}
  {\put(450,80){%
     \makebox(0,0)[rb]{%
       \chapterfont\textcolor[named]{SkyBlue}{\thechapter}}}
   \put(0,230){%
     \makebox(0,0)[lb]{%
       \Huge\sffamily\underline{Chapter \thechapter}}}}
  {0pt}
  {\put(0,190){\parbox[t]{300pt}{%
     \Huge\sffamily\filright#1}}}
  [\end{picture}]
\end{verbatim}

(The exact values to be used depend on the text area,
class, |\unitlength|, paper size, etc.)

\end{document}

